% !TEX TS-program = pdflatex
% !TEX encoding = UTF-8 Unicode

% This is a simple template for a LaTeX document using the "article" class.
% See "book", "report", "letter" for other types of document.

%commento
\documentclass[11pt]{article} % use larger type; default would be 10pt

\usepackage[utf8]{inputenc} % set input encoding (not needed with XeLaTeX)

%%% PAGE DIMENSIONS
\usepackage{geometry} % to change the page dimensions
\geometry{a4paper} % or letterpaper (US) or a5paper or....
% \geometry{margin=2in} % for example, change the margins to 2 inches all round
% \geometry{landscape} % set up the page for landscape
%   read geometry.pdf for detailed page layout information

\usepackage{graphicx} % support the \includegraphics command and options

% \usepackage[parfill]{parskip} % Activate to begin paragraphs with an empty line rather than an indent

%%% PACKAGES
\usepackage{booktabs} % for much better looking tables
\usepackage{array} % for better arrays (eg matrices) in maths
\usepackage{paralist} % very flexible & customisable lists (eg. enumerate/itemize, etc.)
\usepackage{verbatim} % adds environment for commenting out blocks of text & for better verbatim
\usepackage{subfig} % make it possible to include more than one captioned figure/table in a single float
% These packages are all incorporated in the memoir class to one degree or another...

%%% HEADERS & FOOTERS
\usepackage{fancyhdr} % This should be set AFTER setting up the page geometry
\pagestyle{fancy} % options: empty , plain , fancy
\renewcommand{\headrulewidth}{0pt} % customise the layout...
\lhead{}\chead{}\rhead{}
\lfoot{}\cfoot{\thepage}\rfoot{}

%%% SECTION TITLE APPEARANCE
\usepackage{sectsty}
\allsectionsfont{\sffamily\mdseries\upshape} % (See the fntguide.pdf for font help)
% (This matches ConTeXt defaults)
\usepackage{pdfpages}

%%% ToC (table of contents) APPEARANCE
\usepackage[nottoc,notlof,notlot]{tocbibind} % Put the bibliography in the ToC
\usepackage[titles,subfigure]{tocloft} % Alter the style of the Table of Contents
\renewcommand{\cftsecfont}{\rmfamily\mdseries\upshape}
\renewcommand{\cftsecpagefont}{\rmfamily\mdseries\upshape} % No bold!

%%% END Article customizations

%%% The "real" document content comes below...

\title{Memory effect in news spreading networks}
\author{Roberto Bertilone, Francesco Fanchin, Nicola Sella}
%\date{} % Activate to display a given date or no date (if empty),
         % otherwise the current date is printed 

\begin{document}
\maketitle
%LA DIVISIONE IN SECTION è ARBITRARIA, CAMBIATELA PURE. CON L'INGLESE HO FATTO DEL MIO MEGLIO, MI SONO SORPRESO MA A VOI FARà SCHIFO. 
%nON HO TOCCATO PUNTI TECNICI E NEMMENO PROFONDI MISONO BUTTATO SU CIO CHE MI VENIVAEVABE. C'è ANCORA MOLTA STRADA, MOLTE IMMAGINI,
 %UN TUTORIAL CON QUALCHE SIMULAZIONE STUPIDA E ANCORA...O CALNCELLA TUTTO
\section*{Abstract}
We propose to analize the influence of the memory in a news spreading dynamic. In the first part we aim to build a framework of agents 
connected in a network and to equip them with a set of basic functions (or methods) to let arise the dynamical process of diffusion.
The goal is to expose a short presentation of the methodologies used in this agent based model and try to explain it with a simple tutorial.

\section{Introduction}

To analyze and model the interactions between users in the context of news spreading is useful to talk about groups of autonomous agents
linked by friendly ties whose overall views constitute the network.
The the population of the network consists of two breed of agents: sources and users. These two breeds will interact
in autonomous approach during the execution of the program.
The network is initialy a random connections of the agents and modifies this topology following a set of microscopic agent's actions.
Our belief is that the interactions, due to the natural news' diffusion in a social-like network, are guided by the ability of the news
to capture each agent's attention, and by the social influence of the agents themselves.
We hope to observe the spontaneous emergency of the scale-free regime following the dynamical micro-interactions. We also hope to reveal 
the natural segregation behaviour that subjugate a vastity of real social networks.
In addition to these questions We wonder if there can be correlation between diffusion and the lenght of the agents' memory.
To do it, we will build a framework of agents modelling based on a Swarm-like platform named SLAPP\footnote{SLAPP DI PIETRO TERNA BLA BLA USEFUL AND BOH SCARICAILE da github bla bla.}.

\section{Overview}
We have build the model focusing on two cohexisting points of view: its agent based nature and the network framework one. 
We blended these two frames considering a network of agents. 
\subsection{Context}
Focusing on the dynamical process of rumor spreading on the social\footnote{The word \textit{ social} is thinked in a general context} network. Generally, the diffusion of news is studied in a stochastics context, governated by a set of stochastic differetial equations. It arise from a apparent similarity with the epidemiological process. However, the diffusion of epidemies become a viral process when a epidemic threshold is exceeded, instead the rumor spreading process appear a process threshold-less.  The epidemic model of information diffusion is usually a compartimental model in which agents in differents stage coesists in the world.
The majority of initial users stay in the compartment of Ignorants whereas minority of them are the Spreader of the news. There is another compartment, the Stifler, the equivalent of recovery in the SIR model\footnote{Acronym of Susceptible-Infected-Recovery, most famous model of epidemic spreading.}. 
\\ This Spreader-Ignorant-stiffler model (SIs) can to be sketch by a set of simple interactions: one of this is spontanueos indeed the others are binary.
\begin{itemize}
\item$ I \longrightarrow S$
\item $S+I \longrightarrow S + S$

\item $S + S \longrightarrow s + S$

\item $s + S \longrightarrow  s + s$
\end{itemize}

Where we indicate with $S$ the Spreader status, with $I$ the Ignorant and with $s$ the Stifler user. The first process correspond to the spontaneous transition from the ignorant compartment to spreader compartment; the second one correspond to the contamination of a ignorant by contact with a spreader user; the third and fourth one interactions rapresent the transition by contact to the Stiffler compartment.  \\
The development of dynamics is governed by sequences of users transitions from a compartment to another one, until all the users which were spreaders reached the stiffler compartment and the remain ignorant remains in their state.
This approach is applicable to a network of users to predict the reproductive number\footnote{The reproductive number $R_{0}$ is defined by characteristic parameters that affect the diffusion like ke average degree (in first approximation) and the rate of diffusion. } that enable to estimate the future qualitative behaviour of spreading: if this number is above some epidemic threshold then the virality of diffusion is guaranteed.\\
This descrition can to be releable with a single viral diffusion of news, but more difficult would result the analisis by a set of different differential equations when happens that a multiple news travel in the network. If the number of news increase, so that the number of equation increase.\\
 An compartmental approach to study the phenomenon of information diffusion has been widely examineted in the last years, futhermore with very different variants of the native models.
 In literature of spreading of news, only a few of models are build on an agent architecture and we report the references in the bibliography.
There are a varietà di papers che evidenziano dei risultati molto interessanti e particolari riguardo l'influenza sociale, il contagio sociale o ad esempio su homophily and the diffusion across the communities- studi di imfuenza che affect the topology. alcuni studiano la diffusione di bufale o di meme o inseriscono utenti che possono fare factchecking o bot che diffondono o ancora inseriscono le sources d'informazione.

\subsection{Why Agents?}
The agents are a very useful paradigm to model social interactions. 
They can operate in their environment, make decisions and interact each other.
There is no comunication protocol between them. They want to maximize their own benefit which is
communicate and share news with "friends". \\
The environment is not deterministic. Every action can produce different effects in a probabilistic
way, but the single esperiment must to be reproducible.

Also the actions between agents are non deterministic and do not have the complete control over another
agent; the agents have limited senses and sensors.
\\

The required characteristics of each agents are: \begin{itemize}
\item Reactivity: The agents can control the clock and the news spreading around them during the evolution of the system.
\item Proactiveness: Each agent knows what his goal is and how to reach it.
\item Social Ability: They know how to send and recive news, to determine sympathy with neighbours.
\item No mobility is required. 

\end{itemize}

The agents are rationals meaning that they try to reach their goal sympathyzind with their "neighbours"
Each agent can receive information from the world or from another agent; it can also interact with 
the world or with an agent, in order to meet own character or objective.
Their ability to modify the environment is the rimotion of a link in the social network.
The only accessible variable for each agent is the clock number; they can also access something "near"
them.
\\

%### (mostly) reactive agents ###
Each agent makes local fair decisions: global behaviours may emerge.
When active each agent can control the environment and reacto to the changes. It is possible
that during his inactive state the environment has changed and the goal of the agent is unreachable.
For this reason the agent can act in an unpredictable way and somehow irrational.


\subsection{The network of Agents}
The network is compose by Agents eventually connected by a weighted and undirected links.
At the first time, the network is composed by the sum of a fixed number of sources and users. \\The users is linked between them with fixed probability calculated from the average degree desidered and insert by external input. In this way we obtain a random network of users, with exponential trend for the degree distribution in the termodynamic limit.\\ The links are the only possibility to establish a relation between the agent; a random value of weight of the link can to rapresent the previous bias in friendship.\footnote{La tipologia di link scelta evidenzia la bilateralità e l’intensità della comunicazione tra gli agenti.   ....prendendo come esempio uno scambio di informazione tra due persone oppure tra una persona e un giornale, il peso rappresenta il feeling, l’affiatamento pregresso; un segno posto davanti ad esso ne potrebbe specificare l’entità (buono, indifferente, cattivo)..}
 The result of this mechanism of generation of graph not return a real network, because an social real network has the property of scale-free\footnote{The scale free property is described principally by power law degree distribution, sicchè l'andamento per gradi alti tende a zero ma non si annulla. },
 instead random graph have a majority of node-degrees picked around the average-degree. Also, the variance is very large in scale free network and expected the presence of hubs in the network. Though it would be better, we start with a random network with the hope to observe a naturally evolvution in this direction.
\\

The sources are thinkable like a news pool from which the users can to draw news and eventually to spread it. The source are more strongly connected with users that the inner links beween the themeselves users, because we assume that the newspaper are more links that a common user.
The sources contain a fixed number of news (usually three), generate at fixed instant of time and dote of different impact. Each single news is modeling like a vector of topics, which components carry the amount of each topics inside the single news. The news is also equipped with an identified code of the source and with the clock in which appear it in the world. 
Another features of the sources is the innate diversity between them based on the amount of there specific topics in the ''character'' of the source, this ''prevalence of topic'' affect the news at the time of generation. In other words, each news provening by the same source are focused, on average, on the same topics. \\
We try to reproduce the intrinsec diversity of the taste of the users with a vector corrispond to a {\itshape state-of-mind} with dimensionality equal to the number of topics. The state-of-mind of the users is inizialized at random, although in possible further development to be innest a process to changing the state-of-mind during the dynamic flow of time. This State-of-mind rapresent the personality of the user, his interest, and maybe inflence(haahah) own choses.
\\
The users could to diffuse the news in relation to their intention: they have a storage memory and they remember a number of news read and spread in the past.
The users has a some kind of knowledge of his neighbourhood, with whome can to interact, to spread a news, to influence themselves and to do change the state-of-mind. The sources not receive the news from the network: for easier computation also the links between sources and users is indirect even if the news are only in outcome.

The time is scan by the Observer\\
Eh
\\
L’utente avrà inoltre un tempo di attivazione che sarà regolato attraverso un suo orologio interno
tenendo conto del tempo che scorre nel mondo: solo mentre è attivo sarà possibile per lui interagire
con il sistema. 
Ogni fonte genererà nuove notizie tenendo conto lei stessa del
tempo che passa.

We use the SLAPP platform for agent simulation BLa blalbalallalllciaàcià PERO' BISOGNEREBBE PARLARNE ANCHE ALL'INIZIO PER IL VECCHIO ZOZZONE.
Il tempo è scandito dall’observer. 
\section{Model}

Vero e proprio modello
\\


La dinamica del sistema sarà determinata dalle azioni degli utenti; i parametri sopra descritti
influenzeranno le loro decisioni.


Una apposita funzione “distanza” permette di confrontare una notizia con lo stato mentale del
singolo utente; in esso è contenuto un vettore avente come dimensione il numero di topics. Lo stato
contiene inoltre altri parametri che caratterizzano con maggiore specificità il singolo utente, ad
esempio la sua propensione a leggere e diffondere notizie.

\section{Tutorial}
Piccola spiegazione seplificata anche per ebeti


%\begin{figure}[h!]
%\begin{center}
%\includegraphics[scale=1.4]{ex1.PDF}
%\end{center}
%\end{figure}
 Set of environment states

 Set of agent actions

\end{document}
