% !TEX TS-program = pdflatex
% !TEX encoding = UTF-8 Unicode

% This is a simple template for a LaTeX document using the "article" class.
% See "book", "report", "letter" for other types of document.

%commento
\documentclass[11pt]{article} % use larger type; default would be 10pt

\usepackage[utf8]{inputenc} % set input encoding (not needed with XeLaTeX)

%%% PAGE DIMENSIONS
\usepackage{geometry} % to change the page dimensions
\geometry{a4paper} % or letterpaper (US) or a5paper or....
% \geometry{margin=2in} % for example, change the margins to 2 inches all round
% \geometry{landscape} % set up the page for landscape
%   read geometry.pdf for detailed page layout information

\usepackage{graphicx} % support the \includegraphics command and options

% \usepackage[parfill]{parskip} % Activate to begin paragraphs with an empty line rather than an indent

%%% PACKAGES
\usepackage{booktabs} % for much better looking tables
\usepackage{array} % for better arrays (eg matrices) in maths
\usepackage{paralist} % very flexible & customisable lists (eg. enumerate/itemize, etc.)
\usepackage{verbatim} % adds environment for commenting out blocks of text & for better verbatim
\usepackage{subfig} % make it possible to include more than one captioned figure/table in a single float
% These packages are all incorporated in the memoir class to one degree or another...

%%% HEADERS & FOOTERS
\usepackage{fancyhdr} % This should be set AFTER setting up the page geometry
\pagestyle{fancy} % options: empty , plain , fancy
\renewcommand{\headrulewidth}{0pt} % customise the layout...
\lhead{}\chead{}\rhead{}
\lfoot{}\cfoot{\thepage}\rfoot{}

%%% SECTION TITLE APPEARANCE
\usepackage{sectsty}
\allsectionsfont{\sffamily\mdseries\upshape} % (See the fntguide.pdf for font help)
% (This matches ConTeXt defaults)
\usepackage{pdfpages}

%%% ToC (table of contents) APPEARANCE
\usepackage[nottoc,notlof,notlot]{tocbibind} % Put the bibliography in the ToC
\usepackage[titles,subfigure]{tocloft} % Alter the style of the Table of Contents
\renewcommand{\cftsecfont}{\rmfamily\mdseries\upshape}
\renewcommand{\cftsecpagefont}{\rmfamily\mdseries\upshape} % No bold!

%%% END Article customizations

%%% The "real" document content comes below...
%
\title{Memory effect in news spreading networks}
\author{Roberto Bertilone, Francesco Fanchin, Nicola Sella}
%\date{} % Activate to display a given date or no date (if empty),
         % otherwise the current date is printed 

\begin{document}

\maketitle

\section*{Abstract}
Our aim is to analyze the influence of memory in a news spreading dynamic. In order to do that, we have built a framework of agents 
connected in a network and equipped them with a set of basic functions. We wish to observe a diffusion-like process.
In this paper we expose the underlying methodology and try to explain it with a simple tutorial.

\section{Introduction}

For the purpose of modeling the interactions between users in the context of news spreading, it is convenient to talk about autonomous agents
linked by friendly ties whose overall view constitutes the network.
Network population is made of two breeds of agents: sources and users. These two breeds will interact
in autonomous approach during the execution of the program.
The network is initially a random connection of agents and modifies its own topology following a set of microscopic agents' actions.
Our belief is that the interactions, due to the natural news' diffusion in a social-like network, are guided by the ability of the news
to capture each agent's attention, and by the social influence of the agents themselves.
We hope to observe the spontaneous emergency of the scale-free regime following the dynamical micro-interactions. We also hope to reveal 
the natural segregation behaviour that subjugate a vastity of real social networks.
In addition to these questions we wonder if there can be correlation between news spreading and the lenght of the agents' memory.
From a technical point of view, we have worked with the Swarm-like protocol in python 3 named SLAPP\footnote{For reference and download from: https://github.com/terna/SLAPP}.

\section{Overview}
We have built our model focusing on two cohexisting points of view: its agent based nature and the network framework one. 
We have blended these two frames considering a network of agents. 
\subsection{Context}
Let's focus on the dynamical process of rumor spreading in a social\footnote{The word \textit{ social} is thinked in a general context} network.
 News diffusion is generally studied in a stochastic context, ruled by a set of stochastic differential equations.
 There is an apparent similarity with  epidemiological processes. 
However, while epidemic diffusion becomes a viral process when a threshold is exceeded, rumor spreading process seems to be threshold-less.
 The epidemic model of information diffusion is usually a compartimental model in which agents in differents stages coexists in the world.
The majority of initial users stays in the compartment of Ignorants whereas minority of them are the Spreader of news.
 There is another compartment, the Stifler, the equivalent of Recovery in the SIR model\footnote{Acronym of Susceptible-Infected-Recovery, most famous model of epidemic spreading.}. 
\\ This Spreader-Ignorant-stifler model (SIs) can be sketched by a set of simple interactions: one of them is spontanueos indeed the others are binary.
\begin{itemize}
\item$ I \longrightarrow S$
\item $S+I \longrightarrow S + S$

\item $S + S \longrightarrow s + S$

\item $s + S \longrightarrow  s + s$
\end{itemize}

Where we indicate with $S$ the Spreader status, with $I$ the Ignorant and with $s$ the Stifler user. 
The first process corresponds to the spontaneous transition from the ignorant compartment to spreader compartment;
 the second one corresponds to the contamination of an ignorant by contact with a spreader user; 
the third and fourth interactions rapresent the transition by contact to the Stifler compartment.  \\
The development of dynamics is governed by sequences of users transitions from a compartment to another one, until all the users which were spreaders reached the stifler compartment and the remain ignorant remains in their state.
This approach is applicable to a network of users to predict the reproductive 
number\footnote{The reproductive number $R_{0}$ is defined by characteristic parameters that affect the diffusion like ke average degree (in first approximation) and the rate of diffusion. } 
that enable us to estimate the future qualitative behaviour of spreading: 
if this number is above some epidemic threshold then the virality of diffusion is guaranteed.\\
This description can be reliable with a single viral diffusion of news, but when a
multiple news diffusion occurs, the analysis by a set of several differential equations would result more difficult .  
 A compartmental approach to study the phenomenon of information diffusion has been widely examinated in the last years, with very different variants of naive models.
There are also several papers underlining interesting results in social science: for example social influence,infection, segregation, homophily or fake news diffusion; or also the effects of fact-checking or bot-agent insertion in a network.
 In news spreading literature, only few models are built on an agent architecture and we report references in the bibliography.


\subsection{Why Agents?}
Agents are a very useful paradigm to model social interactions. 
They can operate in their environment, make decisions and interact each other.
There is no communication protocol between them. They want to maximize their own benefit which is
communicate and share news with "friends". \\
The environment is not deterministic. 
Every action can produce different effects in a probabilistic way, but the single esperiment must be reproducible.
Actions between agents are non-deterministic too and they don't have complete control of another
agent;agents have limited senses and sensors.
\\

The required characteristics of each agent are: \begin{itemize}
\item Rationality: agents can take choices depending by their own belief and their surrounding environment;
\item reactivity: agents can check the World clock and news spreading nearby during the dynamic;
\item proactiveness:  agents can express their own will, taking autonomous initiatives;
\item social Ability: They know how to send and receive news, to determine sympathy with neighbours;
\item no mobility is required. 


\end{itemize}

Each agent can receive information from the world or from another agent; it can also interact with 
the world or with an agent, in order to meet its own character (a.k.a. state of mind) 
They can  modify their surrounding environment by means of the insertion or removal of a link in the social network.
The only accessible variable for each agent is the clock number; they can also access some neighbours' variables.
\\

Each agent makes local fair decisions: global behaviours may emerge.
When active each agent can control the environment and reacts  to the changes. 
It is possible that during his inactive state the environment has changed so the previous buffered action cannot be performed.
For this reason the agent can act in an unpredictable way and somehow irrational.


\subsection{The network of Agents}
Our network is made of Agents eventually connected by weighted and undirected links.
At the first time, the network is composed by the sum of a fixed number of sources and users. \\
Users are linked between them with fixed probability computed from the desired average degree and inserted by external input.
 In this way we obtain a random network of users, with exponential trend for the degree distribution in the termodynamic limit.
\\ Links are the only possibility to establish a relation between the agents; a random value of weight of the link can rapresent a previous bias in 
friendship.\footnote{This particular choice for our links underlines bilaterality and intensity of communication between agents.
Taking as example an exchange of information between two people or between a person and a newspaper, weight represents feeling, previous chemistry;}
The result of such a mechanism of graph generation doesn't actually return a real network, because social real networks possess the property of 
scale-free\footnote{The scale-free property is mainly described by power law degree distribution, with a cut-off for high degrees (i.e. hubs). }
 ;instead random graphs have a majority of node-degrees peaked around the average-degree. 
Also, the variance is very large in scale-free network and guarantees the existence of hubs in the network.
Despite the abovementioned theory, we start with a random network hoping to observe a natural evolution in this direction.
\\
%%%% fin qua correct
The sources are thinkable like a news pool from which the users can to draw news and eventually to spread it. The source are more strongly connected with users that the inner links beween the themeselves users, because we assume that the newspapers have more links that a common user.
The sources contain a fixed number of news (usually three), generate at fixed instant of time and dote of different impact. Each single news is modeling like a vector of topics, which components carry the amount of each topics inside the single news. The news is also equipped with an identified code of the source and with the clock in which appear it in the world. 
Another features of the sources is the innate diversity between them based on the amount of there specific topics in the ''character'' of the source, this ''prevalence of topic'' affect the news at the time of generation. In other words, each news coming from the same source are focused, on average, on the same topics. \\
We try to reproduce the intrinsec diversity of the taste of the users with a vector corrispond to a {\itshape state-of-mind} with dimensionality equal to the number of topics. The state-of-mind of the users is inizialized at random, although in possible further development to be innest a process to changing the state-of-mind during the dynamic flow of time. This State-of-mind rapresent the personality of the user, his interest, and may affect his own choices.
\\
The users could diffuse the news in relation to their intention: they have a storage memory and they remember a number of news read and spread in the past.
The users has a some kind of knowledge of his neighbourhood, with whome can to interact, to spread a news, to influence themselves and to do change the state-of-mind (campo visivo limitato). The sources not receive the news from the network: for easier computation also the links between sources and users is indirect even if the news are only in outcome.
\\
The time is scan by the Observer\\
Eh
\\
L’utente avrà inoltre un tempo di attivazione che sarà regolato attraverso un suo orologio interno
tenendo conto del tempo che scorre nel mondo: solo mentre è attivo sarà possibile per lui interagire
con il sistema. 
Ogni fonte genererà nuove notizie tenendo conto lei stessa del
tempo che passa.


\section{Our Model}
We use the SLAPP platform for agent simulation,
SLAPP3 provides agent based protocoll in Python 3; specifically: 1observer tempo del mondo - Classi agenti e sui breed(classi derivate) - Usiamo superAgent ch disegna i grafici e crea gli agenti. - azioni generiche gia implementate da specificare a seconda della modellizzazione :DO1.. Draw net... 
\\
Schedule: modo per programmare le varie azioni, in ordine temporale esplicitamente dato d noi.  Le azioni possono essere performate con una certa probabilità... removelinks...
Si interfaccia alle azioni generich sopramenzionate.
\\
Le varie Funzioni base (1.2.3.4.5....  9che concatenate costituiscono le azioni vere e proprie degli agenti: Funzioni e metodi delle classi::: SPECCHIETTO
User/Sources
La dinamica del sistema sarà determinata dalle azioni degli utenti; i parametri sopra descritti
influenzeranno le loro decisioni.
\begin{itemize}
\item Agent.py   (user/source) 
\item SkyAgent  (Message scheduler and ConnectionManager ) write the log files...
\item Graph (Create the Graph/ edges and display the graph)
\item Souces
\item Users

\item msglog
\end{itemize}
Funzioni:
Una apposita funzione “distanza” permette di confrontare una notizia con lo stato mentale del
singolo utente; in esso è contenuto un vettore avente come dimensione il numero di topics. Lo stato
contiene inoltre altri parametri che caratterizzano con maggiore specificità il singolo utente, ad
esempio la sua propensione a leggere e diffondere notizie.

\section{Tutorial}
The basic tutorial 


%\begin{figure}[h!]
%begin{center}
%\includegraphics[scale=1.4]{ex1.PDF}
%\end{center}
%\end{figure}
 Set of environment states

 Set of agent actions

\end{document}
