\section{Tutorial}
Let's see in practice what we have previously learnt in a simple tutorial: first of all, we need the whole program, available at \textit{ https://github.com/BFSteam/memory.}
Once we put it in the same folder containing SLAPP3, we need to connect the two programs inserting memory \textbackslash src path in SLAPP3 \textbackslash project.txt as described in \textit{SLAPP\_Reference\_Handbook.pdf pag20}.
Now, inside SLAPP3, we can run the program runShell.py, for example by terminal.
At the very beginning, we confirm \textit{memory} path and project. Then, we have to set all the necessary input variables to start the simulation:
\begin{itemize}
\item[\texttt{Random number seed:}] insert the seed to provide the entire reproducibility of a single simulation.
\item[\texttt{Number of sources:}]insert the number of sources inside the network.
\item[\texttt{Number of users:}]insert the number of users inside the network.
\item[\texttt{Average degree for users:}]insert the value of the average degree for users only.
\item[\texttt{Number of cycles:}]insert the maximum number time can reaches.
\end{itemize}

In order to simplify our first approach, default values are provided to all previous variables by pressing Enter, and that's it!
Simulation is running and our network is evolving. A window will appear with the drawing network\footnote{The runtime network is drawn using the libraries networkx and matplotlib.} and in terminal the flowing time is displayed. 
Nodes of the network are painted with different colours: if they don't spread, grey for inactive state and blue for active one; if they spread, they can be orange, pink or cyan depending on the spreaded news. In detail, every source initially generates a single news, tracked by its colour. Sources are bigger than users and labeled with 0,1 and 2; numbering goes on with users.\\



- common var
-schedule
-observerActions.txt
networkx