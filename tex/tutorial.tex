\section{Tutorial}
\subsection{Before the simulation: what to do}
Let us see in practice what we have previously learnt in a simple tutorial: first of all, we need the whole program, available at \url{https://github.com/BFSteam/memory.}\\
Once we put it in the same folder containing SLAPP3, we need to connect the two programs inserting \textit{memory \textbackslash src} path in \textit{SLAPP3 \textbackslash project.txt} as described in 
\\ \textit{SLAPP\_Reference\_Handbook.pdf pag20}.
Now, inside SLAPP3, we can run the program runShell.py, for example by terminal.

At the very beginning, we confirm \textit{memory} path and project. Then, we have to set all the necessary input variables to start the simulation:
\begin{itemize}
\item[\texttt{Random number seed:}] insert the seed to make the simulation reproducible.
\item[\texttt{Number of sources:}]insert the number of sources inside the network.
\item[\texttt{Number of users:}]insert the number of users inside the network.
\item[\texttt{Average degree for users:}]insert the value of the average degree for users only.
\item[\texttt{Number of cycles:}]insert the maximum number time can reach.
\end{itemize}

In order to simplify our first approach, default values are provided to all previous variables by pressing Enter, and that's it!\\

\subsection{During the simulation: where to put the accent}
Simulation is running and our network is evolving. A window will appear with the drawing network\footnote{The run time network is drawn using networkx and matplotlib libraries.} and flowing time is displayed in terminal. \\
Nodes of the network are painted with different colors: if they don't spread, gray for inactive state and blue for active one; if they spread, they can be orange, pink or cyan depending on the spread news. In detail, every source initially generates a single news, tracked by its color. Sources are bigger than users and labeled with 0,1 and 2; numbering goes on with users.\\
We can show network dynamic in the pictures below:
\begin{figure}[!h]
  \centering
  \begin{subfigure}[t]{.45\textwidth}
    \centering
    \includegraphics[trim={1cm .5cm 1cm 1cm}, clip, width=\linewidth]{img/pdf/plot-0001.pdf} 
    \caption{1 cycle}
    \label{fig:1}
  \end{subfigure}
  \begin{subfigure}[t]{.45\textwidth}
    \centering
    \includegraphics[trim={1cm .5cm 1cm 1cm}, clip, width=\linewidth]{img/pdf/plot-0005.pdf} 
    \caption{5 cycles}
    \label{fig:5}
  \end{subfigure}

  \vspace{0cm}

  \begin{subfigure}[t]{.45\textwidth}
    \centering
    \includegraphics[trim={1cm .5cm 1cm 1cm}, clip, width=\linewidth]{img/pdf/plot-0010.pdf} 
    \caption{10 cycles}
    \label{fig:10}
  \end{subfigure}
  \begin{subfigure}[t]{.45\textwidth}
    \centering
    \includegraphics[trim={1cm .5cm 1cm 1cm}, clip, width=\linewidth]{img/pdf/plot-0050.pdf} 
    \caption{50 cycles}
    \label{fig:50}
  \end{subfigure}
 
  \caption{Plot at different time steps for a simulation with 3 sources, 200 users, initial average degree of users 3 and 500 time steps. Random seed initialized to 17}
\end{figure}

\begin{figure}
  \centering
  \begin{subfigure}[t]{.45\textwidth}
    \centering
    \includegraphics[trim={1cm .5cm 1cm 1cm}, clip, width=\linewidth]{img/pdf/plot-0100.pdf} 
    \caption{100 cycles}
    \label{fig:100}
  \end{subfigure}
  \begin{subfigure}[t]{.45\textwidth}
    \centering
    \includegraphics[trim={1cm .5cm 1cm 1cm}, clip, width=\linewidth]{img/pdf/plot-0200.pdf} 
    \caption{200 cycles}
    \label{fig:200}
  \end{subfigure}

  \vspace{0cm}

  \begin{subfigure}[t]{.45\textwidth}
    \centering
    \includegraphics[trim={1cm .5cm 1cm 1cm}, clip, width=\linewidth]{img/pdf/plot-0400.pdf} 
    \caption{400 cycles}
    \label{fig:400}
  \end{subfigure}
  \begin{subfigure}[t]{.45\textwidth}
    \centering
    \includegraphics[trim={1cm .5cm 1cm 1cm}, clip, width=\linewidth]{img/pdf/plot-0500.pdf} 
    \caption{500 cycles}
    \label{fig:500}
  \end{subfigure}
 
  \caption{Plot at different time steps for a simulation with 3 sources, 200 users, initial average degree of users 3 and 500 time steps. Random seed initialized to 17}
\end{figure}

\subsection{After the simulation: what to notice}
We can notice that in the fist cycle~\ref{fig:1} news have not spread yet and the agents are mostly inactive. ......................

\subsection{Further implementations}

\subsection{Conclusion}
