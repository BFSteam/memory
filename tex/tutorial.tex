\section{Tutorial}\label{sec:tutorial}
\subsection{Before the simulation: what to do}\label{subsec:before}
Let us see in practice what we have previously learnt in a simple tutorial:
first of all, we need the whole program, available at \url{https://github.com/BFSteam/memory.}\\
Once downloaded it is convenient to put \texttt{path\_to/memory/src} inside
\texttt{path\_to\_SLAPP3/project.txt} as described in \textit{SLAPP\_Reference\_Handbook.pdf pag20}.
Inside SLAPP3 we can run the program \texttt{runShell.py} from terminal
or from jupyter notebook using \texttt{iRunShell.ipynb}.
The program asks which project we want to run: if we set up the path correctly
we can confirm \textit{memory} path and project.
Afterwards we have to set all the necessary input variables to start
the simulation:
\begin{itemize}
\item[\texttt{Random number seed:}] insert the seed to make the simulation reproducible.
\item[\texttt{Number of sources:}]insert the number of sources inside the network.
\item[\texttt{Number of users:}]insert the number of users inside the network.
\item[\texttt{Average degree for users:}]insert the value of the average degree for users only.
\item[\texttt{Number of cycles:}]insert the maximum number time can reach.
\end{itemize}

In order to simplify our first approach, default values are provided:
answering enter at each line will mak the program run, and that's it!\\

\subsection{During the simulation: where to put the accent}\label{subsec:during}
Simulation is running and our network is evolving. A window will appear
with the drawing network\footnote{Graphs drawn using
  \href{https://networkx.github.io/}{networkx} and
  \href{https://matplotlib.org/}{matplotlib}}
and flowing time is displayed in terminal. \\
Nodes of the network are painted with different colors: if they don't
spread, gray for inactive state and blue for active one; if they spread,
they can be orange, pink or cyan depending on the spread news.
In detail, every source initially generates a single news, tracked by its color. Sources are bigger than users and labeled with 0,1 and 2; numbering goes on with users.\\
We can show network dynamic in the pictures below:
\begin{figure}[!h]
  \centering
  \begin{subfigure}[t]{.45\textwidth}
    \centering
    \includegraphics[trim={1cm .5cm 1cm 1cm}, clip, width=\linewidth]{img/pdf/plot-0001.pdf} 
    \caption{1 cycle}
    \label{fig:1}
  \end{subfigure}
  \begin{subfigure}[t]{.45\textwidth}
    \centering
    \includegraphics[trim={1cm .5cm 1cm 1cm}, clip, width=\linewidth]{img/pdf/plot-0005.pdf} 
    \caption{5 cycles}
    \label{fig:5}
  \end{subfigure}

  \vspace{0cm}

  \begin{subfigure}[t]{.45\textwidth}
    \centering
    \includegraphics[trim={1cm .5cm 1cm 1cm}, clip, width=\linewidth]{img/pdf/plot-0010.pdf} 
    \caption{10 cycles}
    \label{fig:10}
  \end{subfigure}
  \begin{subfigure}[t]{.45\textwidth}
    \centering
    \includegraphics[trim={1cm .5cm 1cm 1cm}, clip, width=\linewidth]{img/pdf/plot-0050.pdf} 
    \caption{50 cycles}
    \label{fig:50}
  \end{subfigure}
 
  \caption{Plot at different time steps for a simulation with 3 sources, 200 users, initial average degree of users 3 and 500 time steps. Random seed initialized to 17}
\end{figure}

\begin{figure}
  \centering
  \begin{subfigure}[t]{.45\textwidth}
    \centering
    \includegraphics[trim={1cm .5cm 1cm 1cm}, clip, width=\linewidth]{img/pdf/plot-0100.pdf} 
    \caption{100 cycles}
    \label{fig:100}
  \end{subfigure}
  \begin{subfigure}[t]{.45\textwidth}
    \centering
    \includegraphics[trim={1cm .5cm 1cm 1cm}, clip, width=\linewidth]{img/pdf/plot-0200.pdf} 
    \caption{200 cycles}
    \label{fig:200}
  \end{subfigure}

  \vspace{0cm}

  \begin{subfigure}[t]{.45\textwidth}
    \centering
    \includegraphics[trim={1cm .5cm 1cm 1cm}, clip, width=\linewidth]{img/pdf/plot-0400.pdf} 
    \caption{400 cycles}
    \label{fig:400}
  \end{subfigure}
  \begin{subfigure}[t]{.45\textwidth}
    \centering
    \includegraphics[trim={1cm .5cm 1cm 1cm}, clip, width=\linewidth]{img/pdf/plot-0500.pdf} 
    \caption{500 cycles}
    \label{fig:500}
  \end{subfigure}
 
  \caption{Plot at different time steps for a simulation with 3 sources, 200 users, initial average degree of users 3 and 500 time steps. Random seed initialized to 17}
\end{figure}

\subsection{After the simulation: what to notice}\label{subsec:after}
We show graphs of a previous run: the parameters used are, seed 17, 3 sources,
200 users, initial average degree of 3, and 500 time steps.
In figure~\ref{fig:1} we see the net during initialization. Diffusion
has not started yet and users are mostly inactive:
sources have create one news each.
After five time steps, figure~\ref{fig:5}, some users have changed their state
to active and started to spread.
At thenth iteration, in figure~\ref{fig:10}, almost every user
has a news inside. We have also the first edge creation and remotion.
Now we jump to the fiftieth iteration, figure~\ref{fig:50},to see that
all the users except one are involved in the dynamic. There is still no
clue of segregation.
After another fifty steps we detect the first symptom of segregation.
Moving on of a hundred in a hundred time steps, we see how the orange news
is inclined to disappear.

\subsection{Further implementations}\label{subsec:implementations}
Some possible future features are:
\begin{itemize}
\item [adding and removing nodes during execution] can yield big changes
  in the process;
\item [look for emerging network behaviors] as said in \textit{\nameref{sec:introduction}} we hope to point out scale-free regime;
\item [analyze the activation time] starting from microscopic behaviors
  it is possible to reproduce macroscopic phenomena such as bursty
  patterns\cite{goh_burstiness_2008}
\end{itemize}

\subsection{Conclusion}\label{subsec:conclusion}







