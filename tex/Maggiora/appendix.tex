\appendix
\section{Errorbars for Gini index}

Bootstrap method is used to compute Gini index' mean and error: while for the former the process is pretty straigthforward, this is not the case for the latter.
Average FOUR distribution is asymmetrical and, in general, non-normal; we consider two different types of error.
Given N samples ${x_1,...,x_N}$ whose mean is $\mu$, correct standard deviation is:

$$
\sigma=\sqrt{\frac{\sum_{i=1}^{i=N}(x_i - \mu)^2}{N-1}}
$$


Quantile Standard Error (	QSE), instead, estimates the error by considering the fraction of samples falling within a certain interval: the whole distribution is divided in equal parts by a certain amount of ``quantiles''.\\
For example, in a Normal distribution, the interval $[\mu -\sigma, \mu +\sigma]$ ``covers'' approximately 68\% of the samples(figura...).
In ``quantiles'' terms (percentiles in this case), our errorbar starts from the $17^{th}$ percentile and ends in the $68^{th}$ percentile.\\
For non-normal distributions, standard deviation in general does not follow this property: to preserve it, we can compute QSE with the previous choice of quantiles, just by looking at the cumulative distribution, when values 0.17 and 0.68 are reached.

Immagini di gaussiana con 1 e 2 sigma e cumulativa della gaussiana...

The overall error is divided twofold, ``rightmost'' and ``leftmost''  the mean. For every part of the interval, we pick the ``largest'' between standard deviation and QSE: this conservative approach avoids underestimations in both directions.




\section{Asymmetric errorbars fitting}
The aim of a standard fitting problem is to find a function which reproduces experimental observations.
Let $f_{\theta}$ be the candidate function: $\theta=(\theta_1,...,\theta_m)$ is a vector of m parameters which completely determines function's values.
The optimization is performed with respect to $\theta$ parameters.\\
Given N datapoints with coordinates ${(x_i,y_i)}$ and y-errorbars $\sigma_i$, i=1,...,N, the usual optimization function for weighted interpolation is:

$$
E(\theta)= \sum_{i=1}^{N} \frac{(f_{\theta}(x_i)-y_i)^2}{\sigma_{i}^2}
$$


However, this formula is only applicable for gaussian errors, expressed by the standard deviation: in the problem we are facing, errorbars are not even symmetric.
We have developed an approximate optimization function which counts in this asymmetry.
In a weighted interpolation, the weight is the square inverse of the standard deviation: the ``shortest'' the errorbar, the closest will be $f(x_i)$ to $y_i$.
It might be an idea to split the error in two contributions, to be ``activated'' if the function value is greater or lower than the experimental value.
Let a and b be, respectively, the errorbars ``over'' and ``down'' the mean point: the optimization function for asymmetric errobars interpolation is:
$$ E= \sum_{i=1}^{N} \frac{(f(x_i)-y_i)^2}{a^{2}\mathcal{H}(f(x_i)-y_i)+b^{2}\mathcal{H}(y_i-f(x_i))} $$
$\mathcal{H}(x)$ is the Heaviside function, whose value is zero for negative argument and one for positive argument.\\
\hl{$f(x_i)$ ``sees'' an error of a if over $y_i$, or b if down.}
The optimization \hl{is performed by specific R packages}.