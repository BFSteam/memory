\subsection{Diameter} \label{diameter}
In graph theory, \textit{shortest path} (SP) between node $a$ and $b$ of a graph is the minimum number of edges connecting them (i.e. the minimum number of ``steps'' to go from $a$ to $b$).\\
After computing SP for every couple of nodes in a graph, we can define \textit{diameter} as the maximum of the shortest paths.\\
As with clustering, we ran five  simulations for every memory size, obtaining a plot of diameter's mean over memory size with errors. Final regression will point out possible correlations.
\begin{figure}[h]
  \centering
  \includegraphics[trim={0cm 0cm 0cm 1cm},clip,width=.8\columnwidth]{img/diameter.pdf}
  \caption{diameter's mean-memory size linear weighted fit: five simulations per point.}
  \label{fig:diameter}
\end{figure}
\begin{table}[h] 
\label{tab:clustering_diameter}
\centering
\begin{tabular}{r|cccc}
\toprule
& Slope & Intercept & $R^2$ & $\rho_{xy}$ \\
\midrule
\textit{Clustering} & $(3 \pm 6) \cdot 10^{-4}$ &$(2.8 \pm 0.2) \cdot 10^{-2}$ &$6.8 \cdot 10^{-2}$ & $2.5 \cdot 10^{-1}$ \\
\textit{Diameter} & $(0.3 \pm 1.2) \cdot 10^{-1}$ & $8.2 \pm 0.4$ & $1.8 \cdot 10^{-2}$ & $1.8 \cdot 10^{-1}$  \\
\bottomrule
\end{tabular}
\caption{Fitting results for clustering and diameter.}
\end{table}

