\section{Introduction} \label{introduction}
Rumour spreading is a well known phenomenon in human interactions:
its influence on public opinion\cite{publicoprumsp} and political
elections\cite{politicalrumsp} is studied with different
methodologies, derived from sociology, mathematics,
physics, psicology and computer science.
DK model,\cite{DKmodel} introduced by Dailey and Kendall, was one of
the first attempts to mathematically reproduce the phenomenon by
stochastic differential equations: a famous variant is MK
model.\cite{MKmodel}\\
With the development of complex networks other approaches arise,
including network topology and many stochastic models
like mean-field\cite{meanfield} and interacting markov-chains\cite{IMC} (IMC).
Thanks to the rapid increasement in computing power,
massive computer simulations have been made possible in
most research and industry sectors.\\
Agent-based model (ABM) is a class of computational models for
simulating the actions and interactions of autonomous
agents.\cite{Agentbased}
Rumour spreading, in an ABM context, can be viewed as a network
of interacting agents: rules on single agent's behaviour
will bring, during a computer simulation, macroscopic
effects on the system.\\
This paper represents natural prosecution of the previous work,
''Memory effects in news' spreading networks'' by the same
authors.\cite{ourpaper}\\
While the former described network's structure and agents' actions,
this one clarifies the role of agents' memory length in network
topology and news' distribution.
We hope to find some properties connected with memory size;
we also expect some correlations among information,
spreading and entropy.\\
The entire work required a phenomenological approach as often described in
ABM's literature(citazione): we started from real-world observations trying
to reveal hidden behaviours.
(l'ho messa al passato.. forse hidden behaviours è un po' generico?
si potrebbe usare \hl{underlying})\\
 In the first section the focus will be on two network properties, average clustering coefficient
and diameter. In the second section we will point out the relation
between memory size and Gini index, to take into account inequalities
within new's distribution.
In addition, a phenomenological description of echo chambers
will be provided.
The whole model lacks, for the moment, of a validation with real data.
However, some conclusions can be drawn anyway: memory length
primarily affects news' distribution, while there are no strong
evidencies of an influence in network topology.
Further details may emerge from a real analysis, e.g. collecting data
from social media. Once figured out a suitable way to connect the model with
real data, it will be possible to improve and implement
our theoretical framework. (e se si mettessero queste ultime due frasi
nelle conclusioni invece che qui? \hl{credo che un accenno a
  ci\`o che si pu\`o fare in seguito si possa dire brevemente anche
qui poi magari lo riprendiamo meglio. giusta osservazione})
