\section{Introduction}
Rumour spreading is a well-known phenomena in human interactions: its influence on public opinion and political elections is studied with different methodologies, derived from sociology, mathematics, physics, psicology and computer science.
DK model, (citazione), introduced by Dailey and Kendall, was one of the first attempts to mathematically reproduce the phenomena by stochastic differential equations: a famous variant is MK model.
With the development of complex networks, other approaches included network topology 
and different stochastic models like mean-field (cite) and interacting markov-chains (IMC)(cite).
Thanks to the rapid increasement in computing power, massive computer simulations have been made possible in most research and industry sectors.\\
Agent-based models (ABM) is a class of computational models for simulating the actions and interactions of autonomous agents.
Rumour spreading, in an ABM context, can be viewed as a network of interacting agents: from rules on single agent's behaviour,  a computer simulation will show the effects on the overall system.
This paper represents natural prosecution of the previous work,
''Memory effects in news' spreading networks'' by the same authors.\\
While the former described network's structure and agents' actions, this one clarifies the role of agents' memory length in network topology and news' distribution.
In the first section the focus will be on two network properties, average clustering coefficient and diameter.
In the second section we will point out the relation between memory size
and Gini index, to take into account inequalities within new's distribution. 
The whole model lacks, for the moment, of a validation with real data.

Does memory size have something connected with other properties?
we belive yes and we are expecting to find some correlations related
to information, spreading and entropy. Thus the use of Gini.\\ \\
Does a phenomenological approach make sense? \hl{Yes because abm needs
this type of approach (vedi slides)}\\ \\
Is this model validated with real data? \hl{Not yet.}
\hl{explain bubbles with image}