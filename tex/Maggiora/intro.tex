\section{Introduction}
Rumour spreading is a well known phenomenon in human interactions:
its influence on public opinion \cite{publicoprumsp} and political elections \cite{politicalrumsp} is studied
with different methodologies, derived from sociology, mathematics,
physics, psicology and computer science.
DK model \cite{DKmodel}, introduced by Dailey and Kendall, was one of
the first attempts to mathematically reproduce the phenomenon by
stochastic differential equations: a famous variant is MK model \cite{MKmodel}.\\
With the development of complex networks other approaches arise,
including network topology and many stochastic models
like mean-field \cite{meanfield} and interacting markov-chains \cite{IMC} (IMC).
Thanks to the rapid increasement in computing power,
massive computer simulations have been \hl{made} possible in
\hl{most research and industry sectors}.\\
Agent-based model (ABM) is a class of computational models for
simulating the actions and interactions of autonomous agents \cite{Agentbased}.
Rumour spreading, in an ABM context, can be viewed as a network
of interacting agents: rules on single agent's behaviour
will bring, during a computer simulation, macroscopic
effects on the system \cite{Agentbased}.

This paper represents natural prosecution of the previous work,
''Memory effects in news' spreading networks'' by the same authors \cite{ourpaper}.\\
While the former described network's structure and agents' actions, this one clarifies 
the role of agents' memory length in network topology and news' distribution.
In the first section the focus will be on two network properties, average clustering coefficient and diameter.
In the second section we will point out the relation between memory size
and Gini index, to take into account inequalities within new's distribution.
In addition, a phenomenological description of bubble chambers will be provided.
The whole model lacks, for the moment, of a validation with real data.
However, some conclusions can be drawn anyway: memory length primarily affects 
news' distribution, while there are no strong evidencies of an influence in network topology.
(fine dell' introduzione)


Does memory size have something connected with other properties?
we belive yes and we are expecting to find some correlations related
to information, spreading and entropy. Thus the use of Gini.\\ \\
Does a phenomenological approach make sense? \hl{Yes because abm needs
this type of approach (vedi slides)}\\ \\
Is this model validated with real data? \hl{Not yet.}
\hl{explain bubbles with image}