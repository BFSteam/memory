\begin{abstract}
Agent-based models (ABM) have been widely applied for complex systems modeling.
A set of microscopic rules define agent's autonomous and cooperative actions: at a macroscopic level, the overall system exhibits the so-called ``emergent behaviour".
Rumour speading in networks is a natural environment where ABM could explain complex phenomena like echo chambers.
Generally memory of agents is not taken into account by standard literature: we do consider it in our analysis.
As a starting point, we will rely on our previous work,
''Memory effects in news' spreading networks''.\\
We developed a network framework where agents interact with themselves.
Network's structure and the software codes were
formerly explained: now we will pay particular attention to
the connection between agents' memory length  and some of the
network's statistical properties.
Precisely we will measure, over different memory sizes, average clustering coefficient, diameter of the network and \hl{Gini index} of news' distribution .\\
We will define a measurement procedure based on weighted
regression; then we will suggest a phenomenological description for
each of the measured quantities.
\end{abstract}