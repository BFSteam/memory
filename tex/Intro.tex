\section*{Abstract}
Our aim is to analyze the influence of memory in a news spreading dynamic. In order to do that, we have built a framework of agents 
connected in a network and equipped them with a set of basic functions. We wish to observe a diffusion-like process.
In this paper we expose the underlying methodology and try to explain it with a simple tutorial.
\section{Introduction}

For the purpose of modeling the interactions between users in the context of news spreading, it is convenient to talk about autonomous agents
linked by friendly ties whose overall view constitutes the network.
Network population is made of two breeds of agents: sources and users. These two breeds will interact
in autonomous approach during the execution of the program.
The network is initially a random connection of agents and modifies its own topology following a set of microscopic agents' actions.
Our belief is that the interactions, due to the natural news' diffusion in a social-like network, are guided by the ability of the news
to capture each agent's attention, and by the social influence of the agents themselves.
We hope to observe the spontaneous emergency of the scale-free regime following the dynamical micro-interactions. We also hope to reveal 
the natural segregation behaviour that subjugate a vastity of real social networks.
In addition to these questions we wonder if there can be correlation between news spreading and the lenght of the agents' memory.
From a technical point of view, we have worked with the Swarm-like protocol in python 3 named SLAPP\footnote{For reference and download from: \url{https://github.com/terna/SLAPP}}.

\section{Overview}
We have built our model focusing on two cohexisting points of view: its agent based nature and the network framework one. 
We have blended these two frames considering a network of agents. 
\subsection{Context}
Let's focus on the dynamical process of rumor spreading in a social\footnote{The word \textit{ social} is thinked in a general context} network.
 News diffusion is generally studied in a stochastic context, ruled by a set of stochastic differential equations.
 There is an apparent similarity with  epidemiological processes. 
However, while epidemic diffusion becomes a viral process when a threshold is exceeded, rumor spreading process seems to be threshold-less.
 The epidemic model of information diffusion is usually a compartimental model in which agents in differents stages coexists in the world.
The majority of initial users stays in the compartment of Ignorants whereas minority of them are the Spreader of news.
 There is another compartment, the Stifler, the equivalent of Recovery in the SIR model\footnote{Acronym of Susceptible-Infected-Recovery, most famous model of epidemic spreading.}. 
\\ This Spreader-Ignorant-stifler model (SIs) can be sketched by a set of simple interactions: one of them is spontanueos indeed the others are binary.
\begin{itemize}
\item$ I \longrightarrow S$
\item $S+I \longrightarrow S + S$

\item $S + S \longrightarrow s + S$

\item $s + S \longrightarrow  s + s$
\end{itemize}

Where we indicate with $S$ the Spreader status, with $I$ the Ignorant and with $s$ the Stifler user. 
The first process corresponds to the spontaneous transition from the ignorant compartment to spreader compartment;
 the second one corresponds to the contamination of an ignorant by contact with a spreader user; 
the third and fourth interactions rapresent the transition by contact to the Stifler compartment.  \\
The development of dynamics is governed by sequences of users transitions from a compartment to another one, until all the users which were spreaders reached the stifler compartment and the remain ignorant remains in their state.
This approach is applicable to a network of users to predict the reproductive 
number\footnote{The reproductive number $R_{0}$ is defined by characteristic parameters that affect the diffusion like ke average degree (in first approximation) and the rate of diffusion. } 
that enable us to estimate the future qualitative behaviour of spreading: 
if this number is above some epidemic threshold then the virality of diffusion is guaranteed.\\
This description can be reliable with a single viral diffusion of news, but when a
multiple news diffusion occurs, the analysis by a set of several differential equations would result more difficult .  
 A compartmental approach to study the phenomenon of information diffusion has been widely examinated in the last years, with very different variants of naive models.
There are also several papers underlining interesting results in social science: for example social influence,infection, segregation, homophily or fake news diffusion; or also the effects of fact-checking or bot-agent insertion in a network.
 In news spreading literature, only few models are built on an agent architecture and we report references in the bibliography.


\subsection{Why Agents?}
Agents are a very useful paradigm to model social interactions. 
They can operate in their environment, make decisions and interact each other.
There is no communication protocol between them. They want to maximize their own benefit which is
communicate and share news with "friends". \\
The environment is not deterministic. 
Every action can produce different effects in a probabilistic way, but the single esperiment must be reproducible.
Actions between agents are non-deterministic too and they don't have complete control of another
agent;agents have limited senses and sensors.
\\
The required characteristics of each agent are: \begin{itemize}
\item Rationality: agents can take choices depending by their own belief and their surrounding environment;
\item reactivity: agents can check the World clock and news spreading nearby during the dynamic;
\item proactiveness:  agents can express their own will, taking autonomous initiatives;
\item social Ability: They know how to send and receive news, to determine sympathy with neighbours;
\item no mobility is required. 
\end{itemize}

Each agent can receive information from the world or from another agent; it can also interact with 
the world or with an agent, in order to meet its own character (a.k.a. state of mind) 
They can  modify their surrounding environment by means of the insertion or removal of a link in the social network.
The only accessible variable for each agent is the clock number; they can also access some neighbours' variables.
\\
Each agent makes local fair decisions: global behaviours may emerge.
When active each agent can control the environment and reacts  to the changes. 
It is possible that during his inactive state the environment has changed so the previous buffered action cannot be performed.
For this reason the agent can act in an unpredictable way and somehow irrational.
